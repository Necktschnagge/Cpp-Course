\documentclass[a4paper]{report}
\usepackage[textwidth=17cm, textheight=25cm]{geometry}

\usepackage[utf8]{inputenc}
\usepackage[ngerman]{babel}
\usepackage{amsmath, amssymb}
\usepackage{enumerate}
\usepackage{multicol} % multiple collums in enumerate

\title{C++ Kurs Informatik TU Dresden}
\author{Maximilian Starke}
\date{\today}

\usepackage[thmmarks,amsmath,hyperref,noconfig]{ntheorem} 
  % erlaubt es, Sätze, Definitionen etc. einfach durchzunummerieren.
\newtheorem{satz}{Satz}[section] % Nummerierung nach Abschnitten
\newtheorem{proposition}[satz]{Proposition}
\newtheorem{kor}[satz]{Korollar}

% --------------------------------------math
\theorembodyfont{\upshape}
\newtheorem{beispiel}[satz]{Beispiel}
\newtheorem{bemerkung}[satz]{Bemerkung}
\newtheorem{definition}[satz]{Definition} %[section]

\theoremstyle{nonumberplain}
\theoremheaderfont{\itshape}
\theorembodyfont{\normalfont}
\theoremseparator{.}
\theoremsymbol{\ensuremath{_\blacksquare}}
\newtheorem{beweis}{Beweis}
\qedsymbol{\ensuremath{_\blacksquare}}

%------------------------------------------------------
\usepackage{tikz}
\usetikzlibrary{shapes,arrows}


% Define block styles
\tikzstyle{decision} = [diamond, draw, fill=blue!20, 
    text width=4.5em, text badly centered, node distance=3cm, inner sep=0pt]
\tikzstyle{block} = [rectangle, draw, fill=blue!20, 
    text width=5em, text centered, rounded corners, minimum height=4em]
\tikzstyle{line} = [draw, -latex']
\tikzstyle{cloud} = [draw, ellipse,fill=red!20, node distance=3cm,
    minimum height=3em]
    


\begin{document}
\maketitle
\tableofcontents

\chapter{Einrichtung}

\section{ISO C++}

\subsection{Allgemeines}
Bitte das hier mal noch schonfinkel
\begin{itemize}
\item entwickelt ab 1979 von Bjarne Stroustrup bei AT\&T als Erweiterung der Programmiersprache C
\item von ISO genormt
\item effizient und schnell - Schnelligkeits eines der wichtigsten Designprinzipien von C++
\item hohes Abstraktionsniveau durch unterstützung von OOP
\item ISO Standard beshreibt auch eine Standardbibliothek

\item Paradigmen:
\subitem generisch (Templates)
\item imperativ (Folge von Anweisungen, Gegenteil von Deklarativ siehe Haskell und Logikprogrammierung)
\item objektorientiert (Klassen, Objekte, Vererbung, Polymorphie, Idee: Anlehnung an Realität)
\item prozedural (Begriff mit verschiedenen Bedeutiungsauffassungen, Unterteilung des Programms in Teilstücke / Sinneinheiten)
\item strukturiert (prozedural und Teilung in Sequenz, Verzweigung, wiederholung,.. )
\item funktional (ab C++11, Definitionskleinkram, siehe Wikipedia, Programm als verschachtelter Funktionsaufruf organisierbar)
\end{itemize}

\subsection{Versionen}
C++03
C++11
C++14
C++17

\section{Dateien in einem C++ Projekt}

\begin{center}
\begin{tabular}{|c|c|p{10cm}|}
	\hline
	Dateiendung & Bezeichnung & Inhalt \\
	\hline
	(*.cpp) (*.cc) & Quelldatei & Funktionsimplementation, Klassenimplementation, \newline Berechnungen bzw. eigentliche Arbeit erledigen \\ \hline
	(*.h) & Headerdatei & Funktionsdeklaration, Klassendefinition, \newline Bezeichner öffentlich bekannt machen \\
	\hline
	
\end{tabular}
\end{center}

\begin{tikzpicture}[node distance = 3cm, auto]
    % Place nodes
    \node [cloud] (precompiler) {Precompiler};
    \node [block, left of=precompiler] (source) {*cpp, *.c};
    \node [block, right of=precompiler] (header) {*.h};
    \node [block, below of=precompiler] (precompiled) {cpp without \#precompiler instruction};
%    \node [block, below of=identify] (evaluate) {evaluate candidate models};
%    \node [block, left of=evaluate, node distance=3cm] (update) {update model};
%    \node [decision, below of=evaluate] (decide) {is best candidate better?};
%    \node [block, below of=decide, node distance=3cm] (stop) {stop};
    % Draw edges
%    \path [line] (init) -- (identify);
%    \path [line] (identify) -- (evaluate);
%    \path [line] (evaluate) -- (decide);
%    \path [line] (decide) -| node [near start] {yes} (update);
%    \path [line] (update) |- (identify);
%    \path [line] (decide) -- node {no}(stop);
%	\path [line] (source) -- (precompiler);
%    \node [cloud] (init) {Precompiler};
%    \node [block, left of=init] (expert) {expert};
%    \node [cloud, right of=init] (system) {system};
%    \node [block, below of=init] (identify) {identify candidate models};
%    \node [block, below of=identify] (evaluate) {evaluate candidate models};
%    \node [block, left of=evaluate, node distance=3cm] (update) {update model};
%    \node [decision, below of=evaluate] (decide) {is best candidate better?};
%    \node [block, below of=decide, node distance=3cm] (stop) {stop};
%    % Draw edges
%    \path [line] (init) -- (identify);
%    \path [line] (identify) -- (evaluate);
%    \path [line] (evaluate) -- (decide);
%    \path [line] (decide) -| node [near start] {yes} (update);
%    \path [line,dashed] (system) -- (init);
%    \path [line,dashed] (system) |- (evaluate);
\end{tikzpicture}

\section{Compiler}
\section{IDEs}
\section{Referenzen}
Buch, websites
\section{The Hello World}

\chapter{Datentypen in C++}



\end{document}
