\documentclass[a4paper]{report}
\usepackage[textwidth=17cm, textheight=25cm]{geometry}

\usepackage[utf8]{inputenc}
\usepackage[ngerman]{babel}
\usepackage{amsmath, amssymb}
\usepackage{enumerate}
\usepackage{multicol} % multiple collums in enumerate
\usepackage{hyperref}

\title{C++ Kurs\\TU Dresden\\Fakultät für Informatik}
\author{Maximilian Starke \\ Student der TU Dresden}
\date{\today}

\usepackage[thmmarks,amsmath,hyperref,noconfig]{ntheorem}
\usepackage{listings}




  % erlaubt es, Sätze, Definitionen etc. einfach durchzunummerieren.
\newtheorem{satz}{Satz}[section] % Nummerierung nach Abschnitten
\newtheorem{proposition}[satz]{Proposition}
\newtheorem{kor}[satz]{Korollar}

% --------------------------------------math
\theorembodyfont{\upshape}
\newtheorem{beispiel}[satz]{Beispiel}
\newtheorem{bemerkung}[satz]{Bemerkung}
\newtheorem{definition}[satz]{Definition} %[section]

\theoremstyle{nonumberplain}
\theoremheaderfont{\itshape}
\theorembodyfont{\normalfont}
\theoremseparator{.}
\theoremsymbol{\ensuremath{_\blacksquare}}
\newtheorem{beweis}{Beweis}
\qedsymbol{\ensuremath{_\blacksquare}}

%------------------------------------------------------
\usepackage{tikz}
\usetikzlibrary{shapes,arrows}

    


\begin{document}
\maketitle
\tableofcontents

\chapter{Einrichtung}

\section{ISO C++}

\subsection{Allgemeines}
\begin{itemize}
\item ab 1979 von Bjarne Stroustrup bei AT\&T entwickelt als Erweiterung der Programmiersprache C
\item später von ISO genormt
\vspace{2ex}
\item effizient und schnell - Schnelligkeit eines der wichtigsten Designprinzipien von C++
\item hohes Abstraktionsniveau u.a. durch Unterstützung von OOP
\item ISO Standard beschreibt auch eine Standardbibliothek
\item C++ ist \textbf{kein} echtes Superset von C (siehe stackoverflow.com, \dots)%###
\item C++ ist (wie C) \textbf{case sensitive}
\vspace{2ex}
\item Paradigmen:
	\begin{itemize}
		\item \textbf{generisch} (durch Benutzung von Templates, automatische Erstellung multipler Funktionen für verschiedene Datentypen)
		\item \textbf{imperativ} (Programm als Folge von Anweisungen, Gegenteil von deklarativ siehe Haskell und Logikprogrammierung)
		\item \textbf{objektorientiert} (Klassen, Objekte, Vererbung, Polymorphie, Idee: Anlehnung an Realität)
		\item \textbf{prozedural} (Begriff mit verschiedenen Bedeutungsauffassungen, Unterteilung des Programms in logische Teilstücke (Prozeduren), die bestimmte Aufgaben / Funktionen übernehmen)
		\item \textbf{strukturiert} (prozedural und Teilung in Sequenz, Verzweigung, Wiederholung, \dots )
		\item \textbf{funktional} (ab C++11, Definitionskleinkram, siehe Wikipedia, Programm als verschachtelter [...] Funktionsaufruf 	organisierbar)
	\end{itemize}
\end{itemize}

\subsection{Versionen}
\begin{itemize}
	\item C++98
	\item C++03
	\item C++11
	\subitem wesentliche Neuerungen. Einführung von constexpr, Elementinitialisierer, ... Neue Bedeutung des Schlüsselworts auto \hspace{3cm} \# Referenzen ergänzen
	\item C++14
	\subitem aufweichung der constexpr Bedingungen.
	\item C++17
	\subitem soll 2017 vollendet werden.
\end{itemize}

\section{Dateien in einem C++ Projekt}

\begin{center}
\begin{tabular}{|c|c|p{10cm}|}
	\hline
	Dateiendung & Bezeichnung & Inhalt \\
	\hline
	(*.cpp) (*.cc) & Quelldatei & Funktionsimplementation, Klassenimplementation, \newline Berechnungen bzw. eigentliche Arbeit erledigen \\ \hline
	(*.h) & Headerdatei & Funktionsdeklaration, Klassendefinition, \newline Bezeichner öffentlich bekannt machen \\
	\hline
	(*.o) & Objektdatei & Objektcode (Maschinencode) einer Übersetzungseinheit\\
	\hline
\end{tabular}

\vspace{4ex}

% Define block styles
\tikzstyle{block} = [rectangle, draw, fill=blue!40, 
text width=7em, text centered, rounded corners, minimum height=5em, node distance= 4.5cm, line width = 2pt]


\tikzstyle{cblock} = [rectangle, draw, fill=blue!40, 
text width=7em, text centered, rounded corners, minimum height=5em, node distance= 3.0cm, line width = 2pt]


\tikzstyle{line} = [draw, -latex', line width = 4pt]


\tikzstyle{cloud} = [draw, ellipse,fill=red!40, node distance=3cm, line width = 2pt,
minimum height=3em]

{\large

\begin{tikzpicture}[node distance = 3cm, auto]
    % Place nodes
    \node [cloud] (pc) {Precompiler};
    
    \node [block, left of=pc] (cpp) {*cpp, *.c};
    \path [line] (cpp) --(pc);
    
    \node [block, right of=pc] (h) {*.h};
    \path [line] (h) --(pc);
    
    \node [cblock, below of=pc] (cpp2) {cpp without \#precompiler instruction};
    \path [line, below of = cpp2] (pc) -- (cpp2);
    
    \node [cloud, below of = cpp2] (c) {Compiler};
    \path [line] (cpp2) -- (c);
    
    \node [cblock, below of = c] (o) {*.o};
    \path [line] (c) -- (o);
    
    \node [cloud, below of = o] (l) {Linker};
    \path [line] (o) -- (l);
    
    \node [block, left of =o] (lo) {another\\ *.o};
    \path [line] (lo) -- (l);
    
    \node [block, right of = o] (ro) {another\\ *.o};
    \path [line] (ro) -- (l);
    
    \node [cblock, below of = l] (exe) {*.exe, *.out};
    \path [line] (l) -- (exe);
     
    \node [block, right of = exe] (lib) {*.lib, *.dll};
    \path [line] (l) -- (lib); 

\end{tikzpicture}
}
\end{center}


\section{Compiler}

\begin{tabular}{c|l}
	GCC & \\
	\; gcc & \\
	g++ & \\
	%#####
\end{tabular}

\section{IDEs}
\subsection{JA oder NEIN}
\begin{center}
\begin{tabular}{|c||c|}
	\hline
	\textbf{ohne IDE} & \textbf{mit IDE} \\
	\hline \hline
	Compiler, Linker über Shell bedienen	&	Projekteinstellungen \& Buttons \\
	Texteditor				&		in IDE integriert\\
	evtl. make + makefile	&	automatisch generiertes makefile \\
	Dokumentationen & geordneter Menübaum \\
	\hline \hline
	Einarbeitungszeit(??) & Einarbeitungszeit (??) \\
	für kleine und mittelgroße Projekte & kleine, mittlere und große Projekte \\
	\hline
\end{tabular}
\end{center}

\subsection{IDEs im Überblick}
\begin{center}
\begin{tabular}{|c|c|p{10cm}|}
	\hline
	\textbf{IDE} & \textbf{Plattform} & \textbf{Anmerkungen}\\
	\hline
	Eclipse, Netbeans & Java (JVM) & in und für Java geschrieben, unterstützt auch C++ \\
	Qt SDK & WIN, Linux, Mac & bringt umfangreiches Qt-Framework mit für GUIs u.v.m. \\
	Code::Blocks & WIN, Linux, Mac & \\
	\hline
	Visual Studio & Windows & kostenfreie BVersion für den Hausgebrauch: VS Community 2016 /2017RC, sehr umfangreich (Refactoring Tools, Debugger, Laufzeitanalyse, Frameworks wie MFC, ATL, WTL) und damit auch speicherintensiv, zu installierende Features wählbar, benutzt eigenen MS VC++ Compiler\\
	Orwell DEV-C++ & Windows &\\
	\hline
	Geany & Linux, WIN & schlichter Texteditor mit Syntaxhighlighting und diversen Compile Buttons\\
	KDevlop & Linux, WIN & \# \\
	Anjuta & Linux & \# \\
	\hline
	XCode & MacOS & "`hauseigene"' IDE von Apple\\
	\hline
	
\end{tabular}
\end{center}

\section{Referenzen}

\begin{itemize}
	\item Buch:
	\begin{itemize}
		\item Wolf, Jürgen: C++ - Das umfassende Handbuch. Rheinwerk Computing
	\end{itemize}
	\item Websites:
	\begin{itemize}
		\item \url{http://en.cppreference.com/w/}
		\item \url{ttp://www.cplusplus.com/reference/}
	\end{itemize}
\end{itemize}
\# Anmerkung ergänzen
\section{The Hello World}
\begin{lstlisting}
#include <iostream>

int main(int argc, char* argv[])
// main-Funktion: Einstiegspunkt der Anwendung
// count: Anzahl der uebergebenen Parameter
// arg: Pointer auf ein Array von Pointern auf C-Style-Strings (die Parameter)
// Parameter der main-Funktion duerfen in der Signatur auch weggelassen werden.

// Parameter der main-Funktion 
{   // Beginn vom Anweisungsblock der main-Funktion
	
	std::cout << "Hello World" << std::endl;
	/*
	* implizite Klammerung:
	* ((std::cout) << "Hello World") << (std::endl);
	* std              ... ein Namensraum
	* ::               ... scope-Operator (Bereichsoperator)
	* cout:            ... gepufferter Standardausgabestream
	* <<               ... Ausgabeoperator (auch bitshift-Operator)
	* "Hello World"    ... C-Style-String Literal
	* endl             ... Objekt aus dem std Namensraum,
	                       das einen Zeilenumbruch ('\n') erzeugt.
	* ;                ... Abschluss einer einzelnen Anweisung
	*/
	
	for(int i = 0; i < argc; ++i ){
		std::cout << i << ". Parameter:  " << argv[i] << '\n';
	} // Beipiel fuer die (Verarbeitung &) Ausgabe der Komandozeilenargumente
	
	return 0; // Rueckgabewert 0 "erfolgreich (ohne Fehler) beendet"
}

\end{lstlisting}


\chapter{Datentypen in C++}



\end{document}
